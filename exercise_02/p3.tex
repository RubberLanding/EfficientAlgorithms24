\documentclass{article}
\usepackage{algorithm}
\usepackage{algorithmic}
\usepackage{amsmath}

\begin{document}
\noindent\textbf{a)}\\

\begin{algorithm}
\caption{Cluster Formation with Blackbox Function}
\begin{algorithmic}[1]

\STATE $clusters \leftarrow \emptyset$

\FOR{each $a$ in $P$}
    \STATE $currentCluster \leftarrow \{a\}$
    \FOR{each $b$ in $P$}
        \IF{$d(a, b) \leq r$}
            \STATE $currentCluster \leftarrow currentCluster \cup \{b\}$
        \ENDIF
    \ENDFOR
    \STATE $clusters \leftarrow clusters \cup currentCluster$
\ENDFOR

\STATE $result \leftarrow \text{blackbox}(\mathcal{T} = clusters, N = k)$
\RETURN $result$

\end{algorithmic}
\end{algorithm}

\bigskip  % Adds space between the two sections

\noindent\textbf{b)}\\

This would not be possible since the distance to the cluster center is not known prior to running the algorithm in the standard k-center problem. Therefore there is no metric we could use to assign possible cluster points to the cluster centers.

\bigskip  % Adds more space before the next algorithm
\noindent\textbf{c)}\\


\begin{algorithm}
\caption{Cluster Formation with known OPT}
\begin{algorithmic}[1]

\STATE $clusters \leftarrow \emptyset$

\WHILE{$P \neq \emptyset$}
    \STATE $currentCenter \leftarrow \text{some } c \in P$
    \STATE $currentCluster \leftarrow \emptyset$
    
    \FOR{each $p$ in $P$}
        \IF{$d(c, p) \leq 2OPT$}
            \STATE $currentCluster \leftarrow currentCluster \cup \{p\}$
        \ENDIF
    \ENDFOR
    
    \STATE $clusters \leftarrow clusters \cup currentCluster$
    \STATE $P \leftarrow P \setminus currentCluster$
\ENDWHILE

\RETURN $clusters$

\end{algorithmic}
\end{algorithm}

\end{document}
